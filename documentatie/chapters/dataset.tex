\chapter{Setul de date}

Setul de date pentru această sarcină a fost preluat din cadrul unui
proiect de detectare a efectelor adverse ale medicamentelor de către
DIEGO Lab, din cadrul universtății "Arizona State University". Acestea
au fost preluate făcându-se căutări pe baza numelor de medicamente, după 
care niște experți în domeniul farmaceutic le-au clasificat în cele două
categorii (conțin sau nu conțin reacții adverse ale medicamentelor).

Tweet-urile au fost furnizate pe bază de ID-uri, ele trebuind descărcate
utilizând un script în python 2.*. Lucrând în python 3.* a trebuit să-mi
creez eu acest script (pe baza celui dat). Descărcarea acestora a durat
aproximativ o oră, iar din această cauză le-am salvat local, nemaifiind
nevoie să-l rulez din nou. Din cele 15667 de ID-uri primite, doar 9257 au rămas valabile.
Din acestea 9257, am folosit 70\% din acestea pentru setul de antrenament, iar restul
de 30\% pentrul setul de test.

O mare problemă cu acest set de date este faptul că este o mare diferență dintre
numarul de date de clasă 0 si numărul de date de clasă 1 (datele sunt nebalansate).
