\chapter{Preprocesarea limbajului}

Primul pas pe care l-am folosit pentru antrenarea modelului a fost preprocesarea textului.
Deoarece textele noastre, ca date de antrenare și de testare, sunt postări de pe \href{https://twitter.com/home}{Twitter},
ele vor avea un limbaj informal, cuvinte prescurtate, argouri, ceea ce face antrenarea modelului nostru
dificilă. Ca urmare a acestui fapt, am preprocesat textul pentru a-l aduce într-o formă relativ mai ușoară 
de înțeles pentru calculator.

Primul pas, alături de cel de-al doilea, au fost inspirați din lucrarea aceasta, aceștia constând în 
înlocuirea unor termeni specifici (medicamentele și reacțiile adverse) cu un simbol, deoarece acestea nu sunt
relevante în clasificarea tweet-urilor. Am folosit o listă
de medicamente și de reacții adverse de pe situl celor de la \href{http://diego.asu.edu/Publications/ADRClassify.html}{Diego Lab}.
Am înlocuit medicamentele cu simbolul "MED", iar reacțiile adverse cu "ADR".

Următorul pas, cel de-al treilea, a fost înlocuirea cuvintelor prescurtate, de tipul "i'm", "you're", în cuvinte întregi,
pentru a restrânge numărul cuvintelor totale, dar și pentru a le da un sens mai puternic acestora.

Al patrulea pas, unul care nu este important de unul singur, dar prinde valoare datorită următorilor pași, este
cel de înlocuirea url-urilor cu un simbol specific. Url-urile având caractere speciale (":", "/") care trebuie și 
ele eliminate din text, se vor crea cuvinte nefolositoare, de exemplu "http". Simbolul folosit pentru acestea este "LINK".

Al cincilea pas constă în înlocuirea referințelor către alte persoane, de forma "@cineva", într-un simbol specific ("REF"),
deoarece acestea conțin diferite nume, care nu au nicio importanță.

Ultimul pas cuprinde eliminarea din text a majoritatății caracterelor non-alfanumerice pentru a mai aerisi textul și a despărți unele cuvinte de acestea,
existând situații când aceste caractere, fiind lipite de cuvinte, construiesc altele noi.