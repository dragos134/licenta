\chapter*{Contribuții} 
\addcontentsline{toc}{chapter}{Contribuții}

În antrenarea modelului dorit a trebuit să folosesc mai multe tehnologii deja existente, pe care
le voi enumera în acest capitol.

Pentru început, ca limbaj de programare am folosit Python 3.8, deoarece există foarte multe librării din domeniul de învățare 
automată. Pe deasupra, acestea, ca toate librăriile din Python, se instalează foarte ușor folosind-une de modului pip. Așadar,
am foloist următoarele biblioteci:
\begin{itemize}
    \item scikit-learn
    \item gensim
    \item fasttext
    \item tensorflow
    \item keras
\end{itemize}
, pe lângă librariile standard.


Pentru modelarea limbajului am folosit următoarele tehnici:
\begin{itemize}
    \item n-grame
    \item Word2Vec, din cadrul librăriei gensim
    \item fasttext, din cadrul librăriei fasstext
\end{itemize}

În final, ca și clasificatori, am folosit următorii algoritmi:
\begin{itemize}
    \item SVM
    \item regresie logistică
    \item bayes naiv
    \item rețele neuronale
\end{itemize}

În următoarele capitole, voi prezenta aceste tehnici în amănunt.