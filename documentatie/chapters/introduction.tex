\chapter*{Introducere} 
\addcontentsline{toc}{chapter}{Introducere}

Problema reacțiilor adverse ale medicamentelor este una de mare notorietate 
în lumea medicinală. Cei care au cel mai mult de suferit și de pierdut de pe
urma acestei probleme sunt consumatorii acestora, aceștia putându-și pune viața
în pericol în anumite cazuri (prescripții greșite, nerespectarea prescripției),
însă afectate sunt și companiile, producătorii și toți cei care lucrează în sfera
farmaceuticelor. 

Reacțiile adverse ale medicamentelor înseamna orice simptome neprevăzute, malicioase
pe care pacientul le simte după un anumit interval de timp de la ingerarea medicamentului.
Pentru a se preveni această problemă, se efectuează seturi de teste, pe pacienți care se
înscriu voluntar la această acțiune, însă, există cazuri în care aceste teste nu sunt
suficiente, iar la momentul lansării medicamentului, oamenii să experimenteze reacții
adverse. Acest fapt este bine cunoscut de toată lumea, de aceea se colectează mereu
date despre efectele medicamentelor și după lansare. Colectarea se dovedește a fi
destul de minuțioasă, deoarece nu toți pacienții vorbesc cu doctorii lor despre aceste
lucruri, dar și din alte motive. Ideea este ca nu există o metodă standardizată pentru
colectarea datelor referitoare la efectele acestor medicamente.

S-a observat că unii pacienți preferă să-și exprime nemulțumirea, alături de efectele
adverse ale medicamentelor, în mediul online. Dar, desigur că și de aici este anevoios
de extras anumite informații, din cauza unor exprimări precare, utilizarea argourilor, etc.
La Workshop-ul AMIA-2017, prin intermediul unui concurs, s-a încercat rezolvarea acestei
ramuri a problemei. La concurs au putut participa echipe din toată lumea, iar în total au
fost desemnate trei probleme. Eu în această lucrare mă voi axa doar pe una, cea descrisă mai sus.

Pentru rezolvarea acestei probleme, am încercat antrenarea unui model care să aibă în final 
un scor cât mai mare la testare, inspirându-mă din lucrarea de \href{https://arxiv.org/pdf/1805.04558.pdf}{aici}. 
Am încercat prin două metode de transformare a textului 
într-un format ușor de înțeles de către calculator (transformând în vectori de numere),
acestea fiind n-grame-ele, care s-au dovedit a nu da un scor foarte bun, și transformarea
word2vec, combinată cu un algoritm SVM din biblioteca sci-kit learn, care a dat și rezultatele
cele mai bune.

